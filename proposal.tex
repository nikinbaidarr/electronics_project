
\documentclass{article}

\usepackage[a4 paper, top=1in, left=1in,right= 1in, bottom=0.5in,
footskip=1in]{geometry}

\usepackage[english]{babel}
\usepackage[utf8x]{inputenc}
\usepackage{amsmath}
\usepackage{graphicx}
\usepackage{wrapfig}
\usepackage{enumitem}
\usepackage{tocloft}
\usepackage{listings}
\usepackage{filecontents}
\usepackage{verbatim}
\usepackage{eurosym}
\usepackage[export]{adjustbox}
\usepackage[hidelinks,linktoc=all]{hyperref}
\usepackage{xcolor}
\usepackage{float}

%% Configs:

\setlength{\parindent}{0em}
\setlength{\parskip}{1.5em}

\graphicspath{{.img/}}

\newcommand{\HRule}{\rule{\linewidth}{0.1mm}}

\begin{document}

\begin{titlepage}
\begin{center}

  \textsc{\huge Purbanchal University}\\[1cm]
  {\huge College of Biomedical Engineering and Applied Sciences}\\[1cm]
  % \textsc{\large \bfseries Tissue Device Interactions}\\[0.8cm]
  % {\large [BEG 3B2 BM]}\\[0.5cm]
  \includegraphics[width=0.4\textwidth]{../cbeas-logo.png}\\[1cm]

  \color{red} \HRule \\[0.4cm] \color{black}
  {\huge \bfseries Application of Machine Learning Algorithms for the
  detection of brain tumours in images acquired through Magnetic
  Resonance Imaging}\\[0.2cm]
  \color{red} \HRule \\[2cm] \color{black}

% Author and supervisor
\textbf{
\Large Proposed By:\\
\Large Lucky Chaudhary [A27]\\ Namrata Tamang [B3]\\ Nikin Baidar
  [B4]\\ Nilima Sangachchhe [B5]\\ Shashwot Khadka [B18]\\ Sneha
  Khadka [B22]\\Suhana Chand [B23]\\[1cm]}
% \Large Supervisor: \\[0.2cm]
% \Large Prof. Alaka Acharya??}
\vfill
{\Large June 2021.}

\end{center}
\end{titlepage}


  \iffalse
    % point

    market ma mental disorder ko lagi widely accepted kunai pani
    imaging wa molecular basis chaina

    research haru bhako cha tara reverse fallacy le garda they are
    rather believed to be useless and inaccurate.

    tara advancements haru bhairacha and hamro reasearch le tyo
    advancement ma contribute garcha

  \fi

\begin{abstract}

  The absence of biological markers makes it exceptionally difficult
  for neurologists to diagnose a person with a mental disorder.
  Currently, diagnosis of mental disorders is based on behavioral
  observations and patient-reported symptoms and the Diagnostic and
  Statistical Manual of Mental Disorders (DSM) classification.

  Although there have been studies that implement imaging techniques
  for deciphering the etiology and the physical cause of several
  mental disorders, the findings from brain imaging do not appear
  amongst the diagnostic criteria. This essentially means that
  neuroimaging is not widely accepted in the process of psychiatric
  diagnosis. The primary reason for this is reverse fallacy.

  Nonetheless, a defiant minority now have started to implement
  neuroimaging techniques such as fMRI, SPECT, PET for the diagnosis
  psychiatric disorders, however, there are no solid molecular or
  imaging basis that are widely accepted for the assessment of mental
  disorders.

  % fMRI ko barema halka

  Here in the proposed research we will be \textit{assessing} MR
  images of 35 subjects who, are suffering or have suffered, from one
  major depressive disorder and \textit{making an attempt} at arriving
  to a comprehensive conclusion about how the ``limbic brain network''
  of patients suffering from Major Depressive Disorder compare to that
  of healthy individuals who share similar socio-demographic
  parameters as the subjects.

\end{abstract}

\newpage
\section{Introduction}

Major Depressive Disorder is one of the most commonly diagnosed mental
disorder in the entire world.

MDD has similarities with schizophrenia and bipolar disorder.

diagnosis based on symptoms and behavioral observations.

\newpage

\section{Objectives}

The objectives of the proposed project are as follows:

\subsection{General Objectives}

\begin{itemize}

  \item Deploy computational tools such as the Independent component
    analysis, and develop image processing strategies for the
    exploration of MR image datasets of brain using AFNI.

  \item Explore data visualization tools, with emphasis on displaying
    functional brain networks.

  \item To perform Seed-based Analysis (SCA) to explore functional
    connectivity within the brain. based on the time series of a seed
    voxel or Region of Interest (ROI).

  \item To perform Seed-based Analysis (SCA) to explore functional
    connectivity of a specific region of interest within the brain and
    see how it connects to the rest of the brain.

\end{itemize}

% to derive conclusions for the sake of validating the statement
% "psychological symptoms and behavioral defects in patients suffering
% from mental health issues are closely related to structural and
% functional changes in specific regions of the brain".

\subsection{Specific Objectives}

\begin{itemize}

  \item Perform analysis of the functional and structural connectivity
    of the hippocampal network of patients suffering from Major
    Depressive Disorder and acquire a comprehensive idea about how it
    compares to that of normal individuals from the same
    socio-demographic background.

\end{itemize}

% public datasets including those involving
% patient outcomes, relating pathology to cellular expression (RNA seq)
% and genetic phenotypes, among others.

% \begin{itemize}
  % \item The functional and structural connectivity of Hippocampal
    % network in MDD : To study of how structure and function of brain
    % with MDD is different than that of normal. : Study the effect of
    % MDD in hippocampus of patients suffering from MDD
% \end{itemize}

\newpage

\section{Problem Statement}

\subsection{Need For An Imaging Basis}

The diagnosis procedures that are the gold standard for diagnosis of
psychiatric disorders are wholly based on behavioral observations and
patient reported symptoms. There are two most widely established
symptoms that are used to classify these manifestations, one is the
Diagnostic and Statistical Manual of Mental Disorders (DSM) and the
other is International Classification for Diseases (ICD).

Despite each being as widely used as the other, both of these
diagnosis manuals are more like frameworks provide a way of
classifying a psychiatric disorder depending on patterns of behaviour
rather than interpreting the etiology and the physical cause of those
disorders.

This statement alone raises an argument that ``although reliable,
current diagnostic procedures in psychiatry are not entirely valid''.

\iffalse
Reliable in the sense that any trained professional will arrive at
the same diagnosis for each patient.

Valid in the sense that it reflects the underlying psychological
and biological commonalities and differences among different
disorders to a certain extent. Validity continues to be more
difficult to achieve.
\fi

Let us take an example of the diagnostic procedure involved in the
diagnosis of Major Depressive Disorder. The DSM-V, published in 2013,
is the most up-to-date manual and is based upon the work of expert
study groups and makes use of large sets of data. According to the
DSM-V, for a person to be classified as "suffering from Major
Depressive Disorder", \textit{he/she must report with either depressed
mood or anhedonia (inability to feel pleasure in normally pleasurable
activities) along with four out of eight additional symptoms}.

This makes it totally possible for 2 distinct individuals who do not
share a single symptom in common and yet receive treatment (or
medication) for MDD.

Furthermore, the current diagnostic procedures such as the DSM-V are
not entirely bulletproof. For example, impulsivity, emotional lability
(the property of changing rapidly), and difficulty with concentration
each occurs in more than one disorder.

Now, the fact that,

\begin{enumerate}[nosep]
  \item Different exemplars of the same category can share no
    symptoms and that
  \item The exemplars of two different categories may share common
    symptoms
\end{enumerate}

raises questions about the validity of the current diagnostic
procedures in psychiatry. Therefore, an imaging basis is necessary for
the diagnosis of mental disorders.

WHY IMAGING WON'T WORK?

There exists thousands of published research studies using
functional neuroimaging methods such as SPECT, PET, and fMRI.

Findings from brain imaging do not appear amongst the diagnostic
criteria. aside from its use to rule out potential physical causes
of a patient's condition, for example a brain tumor, neuroimaging
is not used in the process of psychiatric diagnosis.

Why has diagnostic neuroimaging not yet found a place in
psychiatric practice? Sensitivity, specificity and standardization
in psychiatric brain imaging.

    - Diagnosis must be made for individuals and not groups
    - meta-analyses of neuroimaging studies has yet to reveal
      patterns of neural activity that are unique to specific
      mental disorder

Sensitivity

Imaging studies are generally not highly sensitive to the
difference between illness and health

Specificity:

Most psychiatric imaging studies involve subjects from only two
categories- patients from a single diagnostic category and people
without any psychiatric diagnosis (healthy individuals), the most
that can be learned from such a study is how brain activation in
those with a particular disorder differs from brain activation in
those without a disorder.

This raises a dilemma for the diagnosing clinician, as the
question is not "does this person have disorder X or is she
healthy? " but "does this patient have disorder X,Y,Z or is she
healthy?" because the pattern of images that distinguishes
patients with disorder X from healthy people may not be unique to
X but shared with a whole alphabet of other disorders.

For example, Amygdala ko example

- (more sophisticated) methods of image analysis may hold promise
  discerning the underlying differences among the many disorders
  that feature similar regional abnormalities !??


- ``statistical approach'' to image analysis makes it possible to
  discover, \underline{spatial and temporal} patterns that
  correspond to performance of specific tasks and specific
  diagnoses. Such statistical methods have only been begun to be
  applied to clinical disorders but show promise for increasing
  the ``specificity'' of brain imaging markers for mental illness.


near-term and long-term prospects of neuroimaging? and what
obstacles block the use of such methods? Answer to the 2nd: The
nature of imaging studies and of psychiatric diagnosis.

Standardization

Standardization is relevant in the sense that protocols for
imaging studies differ from study to study, particularly amongst
functional imaging studies.

The results of psychiatric imaging research are often summarized
by stating that certain regions are under or over active or more
or less functionally connected.

Findings on the patterns of activation acquired in studies of
psychiatric patients depends strongly on the task being performed
by the subjects and the statistical comparisons made by the
researcher afterwards. Such findings are pretty much incomplete
unless they include the information about what task evoked the
activation in question: whether the patient wa resting, processing
an emotional stimuli, resisting emotional stimuli or engaged in
some other task?

Therefore the fact that imaging study's conclusions are relative
to the tasks performed adds further complexity to the problem of
consistently discriminating patterns of activation of healthy and
ill subjects.

\newpage
\section{Review of Literature}

% opening note:

Implementation of fMR-imaging techniques to research the core aspects
of structural and functional brain alterations in patients suffering
from MDD.

Past work seems like structural MRI and fMRI look promising for
providing excellent and reliable indexes for the aid in the diagnosis
and ultimately treatment of MDD


% ending note:

trar ahile samma figure out bhako kura haru lai chai actual practice
ma lyauna sakeko chaina

\end{document}
