
\documentclass{article}

\usepackage[a4 paper, top=1in, left=1in,right= 1in, bottom=1in,
footskip=0.5in]{geometry}

\usepackage[english]{babel}
\usepackage[utf8x]{inputenc}
\usepackage{amsmath}
\usepackage{graphicx}
\usepackage{wrapfig}
\usepackage{enumitem}
\usepackage{tocloft}
\usepackage{listings}
\usepackage{filecontents}
\usepackage{verbatim}
\usepackage{eurosym}
\usepackage[export]{adjustbox}
\usepackage[hidelinks,linktoc=all]{hyperref}
\usepackage{xcolor}
\usepackage{float}

%% Configs:

\setlength{\parindent}{0em}
\setlength{\parskip}{1.5em}

\graphicspath{{.img/}}

\newcommand{\HRule}{\rule{\linewidth}{0.1mm}}

\begin{document}

\begin{titlepage}
\begin{center}

  \textsc{\huge Purbanchal University}\\[1cm]
  {\huge College of Biomedical Engineering and Applied Sciences}\\[1cm]
  % \textsc{\large \bfseries Tissue Device Interactions}\\[0.8cm]
  % {\large [BEG 3B2 BM]}\\[0.5cm]
  \includegraphics[width=0.4\textwidth]{../cbeas-logo.png}\\[1cm]

  \color{red} \HRule \\[0.4cm] \color{black}
  {\huge \bfseries Seed-based Analysis of Functional Connectivity of
  Hippocampal Network of People Suffering from Clinical Depression}\\[0.2cm]
  \color{red} \HRule \\[2cm] \color{black}

% Author and supervisor
\textbf{
  \Large Proposed By:\\[0.2cm]
\Large Lucky Chaudhary [A27]\\ Namrata Tamang [B3]\\ Nikin Baidar
  [B4]\\ Nilima Sangachchhe [B5]\\ Shashwot Khadka [B18]\\ Sneha
  Khadka [B22]\\Suhana Chand [B23]\\[1cm]}
% \Large Supervisor: \\[0.2cm]
% \Large Prof. Alaka Acharya??}
\vfill
{\Large June 30, 2021}

\end{center}
\end{titlepage}

\pagenumbering{roman}
\clearpage
\setcounter{page}{1}

% abstract
\begin{center}
  \textbf{\large Abstract}
\end{center}

  The absence of biological markers makes it exceptionally difficult
  for neurologists to diagnose a person with a mental disorder.
  Currently, diagnosis of mental disorders is based on behavioral
  observations and patient-reported symptoms and the Diagnostic and
  Statistical Manual of Mental Disorders (DSM) classification.

  Although there have been thousands of studies revolving around
  implementation of various imaging modalities for deciphering the
  etiology and the physical cause of several mental disorders, the
  findings from these studies do not appear amongst the diagnostic
  criteria. Meaning that the findings from these studies are not used
  for diagnosis purposes. A critical barrier to the clinical
  translation of many findings is the reverse inference fallacy.

  Reverse inference is a kind of reasoning that is applied to infer
  the involvement of a specific cognitive process from observed brain
  activation during a task.  It attempts to uncover specific cognitive
  processes or behaviours that may be associated with specific
  structural or functional brain alterations. However, reasoning
  backwards from brain activity is problematic because neurological
  disorders are multifaceted and are influenced by several factors
  such as concurrent diseases, disease history and artifacts.

  For this reason alone, neuroimaging is not ``widely accepted'' in
  the process of psychiatric diagnosis.  Despite, the reverse
  inference fallacy, neuroimaging for diagnosis of mental disorders
  seems promising in the future and as a matter of fact, a bold
  (choose correct word here) minority have already started to
  implement neuroimaging techniques such as fMRI, SPECT, PET for the
  diagnosis psychiatric disorders. Nevertheless, there are no solid
  molecular or imaging basis that are widely accepted for the
  assessment of mental disorders.

  Here in the proposed research we will be \textit{assessing} MR
  images of 35 subjects who, are suffering or have suffered, from one
  major depressive disorder and \textit{making an attempt} at arriving
  to a comprehensive conclusion about how the ``limbic brain network''
  of patients suffering from Major Depressive Disorder compare to that
  of healthy individuals who share similar socio-demographic
  parameters as the subjects.

\newpage

\pagenumbering{arabic}
\clearpage
\setcounter{page}{1}

\section{Introduction}

\subsection{Major Depressive Disorder}

Major Depressive Disorder, generally abbreviated as MDD, is one of the
most common and a serious mental disorder. MDD is also referred to as
clinical depression, or just depression as well.

MDD can be characterized by an array of distinct symptoms; persistent
felling of sadness, feelings of low self-worth and guilt, and an
overall reduced ability to take pleasure from activities that
previously were enjoyable are a few symptoms prevalent in MDD.
Although, the exact symptoms of depression may vary from person to
person, depending on their upbringing and various socio-demographic
variables such as age, sex, religious affiliations, employment, income
etc; for an individual to be classified as ``suffering from MDD'',
five out of ten symptoms, one from a set of two and % at least four
additional symptoms from another set of 5, must have to % have been
present as a bare minimum during span of 2 weeks.  % insert % WHO
reference here

In addition to the symptoms that may be prevalent in a person
suffering from depression, there can be morphological differences in
several brain regions, including the frontal and temporal lobes. On
top of that, individuals suffering from  also have abnormal functional
connectivity. % which we will be exploring in this project.

According to the World Health Organization, more than 264 million
people of all ages suffer from depression worldwide. Fortunately,
there are effective psychological and pharmacological treatments for
moderate to severe depression. The pharmacological treatment includes
medications, SSRIs and SNRI are two antidepressants that are most
commonly prescribed. The psychological treatments include
psychotherapy and electroconvulsive therapy depending on the severity
of the depression, treatment can take a few weeks or much longer.

\subsection{Brain Networks}

A brain network, on a large scale, can be defined as a collection of
brain regions working together to produce a specific function.

Brain networks can be identified at various different resolutions,
therefore there is no universal atlas of brain networks that fits all
circumstances. However, on the basis of converging evidences from
related studies, there are six large-scale, core brain networks that
are most widely accepted due to their stability:

\begin{enumerate}[nosep]
  \item Default Mode Network
  \item Salience Network
  \item Dorsal Attention Network
  \item Frontoparietal Network
  \item Sensorimotor Network
  \item Visual Cortex
\end{enumerate}

There are more subsets of these six networks such as the limbic,
auditory, right/left executive, cerebellar, spatial attention,
language, lateral visual, temporal and visual perception/imagery.

An emerging paradigm in neuroscience is that cognitive tasks are
performed not by individual brain regions working in isolation but
rather by brain networks consisting of several discrete brain regions
that are said to be ``functionally connected''. The functional
connectivity of brain networks can be acknowledged through
(statistical) analysis of images acquired through a variety of
techniques such as the fMRI, EEG, PET or SPECT.

% Statistical analysis makes it possible to discover, spatial and
% temporal patterns that correspond to performance of specific tasks and
% specific diagnoses. Statistical methods have only been begun to be
% applied to clinical disorders but show promise for increasing the
% ``specificity'' of brain imaging markers for mental illness such as
% the MDD.

\subsection{Resting State Functional Connectivity}

Resting-state functional connectivity can be defined as a %
significant correlated signal between functionally connected brain
regions in the absence of any stimulus or task. Resting-state
functional connectivity measures \textbf{temporal correlation} of
spontaneous Blood Oxygen Level Dependent (BOLD) signal among spatially
distributed brain regions, with the assumption that regions with
correlated activity form functional networks.

There are two methods that are most commonly used to examine
functional connectivity:
\begin{itemize}[nosep]
  \item Seed-based Correlation Analysis (SCA) and
  \item Independent Components Analysis (ICA)
\end{itemize}

In seed-based approaches, activity is extracted from a specific region
of interest and correlated with the rest of the brain. In contrast,
ICA does not begin with pre-defined brain regions. It is a
multivariate, data-driven approach that deconstructs fMRI time-series
data throughout the brain into separate spatially independent
components.

The resting state fMRI produces reliable and reproducible results, and
in addition to that, there are several features of resting-state fMRI
that it makes it favorable for investigating the correlation of
psychiatric and neurological disorders.  First, compared to the
modular representations of traditional fMRI, functional connectivity
provides a broader network representation of the functional
architecture of the brain. Second, the absence of an explicit task
eases the \textit{cognitive demand of the fMRI environment}, thereby
eliminating the problem of whether or not to match groups on task
performance and allowing researchers to investigate under-studied
populations, including infants and cognitively impaired individuals.
Finally, the relatively standard manner in which resting-state fMRI
data are acquired makes it ideal for multi-site investigations and
data sharing.

\subsection{Neuroimaging}

Neuroimaging or brain imaging is the use of various techniques to
either directly or indirectly image the structure, function, or
pharmacology of the nervous system. Current neuroimaging techniques
reveal both form and function. They reveal the brain's anatomy,
including the integrity of brain structures and their
interconnections. Neuroimaging falls into two broad categories:

\begin{enumerate}

  \item Structural imaging, which deals with the structure of the
    nervous system and the diagnosis of gross (large scale)
    intracranial disease (such as a tumor) and injury.

  \item Functional imaging, which is used to diagnose metabolic
    diseases and lesions on a finer scale (such as Alzheimer's
    disease) and also for neurological and cognitive psychology
    research and building brain-computer interfaces.

\end{enumerate}

\enlargethispage{\baselineskip}
Functional   Magnetic   resonance   imaging   (fMRI), a modern
technique of imaging, is a powerful non-invasive and safe tool which
is used for the study of the function of the brain based on measure of
the brain neural activation (Farah, 2002). The fMRI can localize the
lo-cation  of  activity in the  brain  which  is  caused  due  to
sensory  stimulation  or  cognitive  function (Gabral,  Sil-veira, \&
Figueredo, 2011). In clinical field, fMRI allows the researchers to
study how are the healthy brain func-tions, how different diseases
affect the brain functions, how  healthy  brain  function  is
recovered  after  damage  and  how  drugs  can  control  the  diseases
effect  on  the  brain activity (Daliri \& Behroozi, 2012).  near-term
and long-term prospects of neuroimaging? and what obstacles block the
use of such methods? Answer to the 2nd: The nature of imaging studies
and of psychiatric diagnosis.

\newpage

\section{Objectives}

The objectives of the proposed project are as follows:

\subsection{General Objectives}

\begin{itemize}

  \item Deploy computational tools , and develop image processing
    strategies for the exploration of MR image datasets of brain using
    AFNI.

  \item Explore data visualization tools, with emphasis on displaying
    functional brain networks.

  \item To perform Seed-based Analysis (SCA) to explore functional
    connectivity within the brain based on the time series of a seed
    voxel or Region of Interest (ROI).

\end{itemize}

\subsection{Specific Objectives}

\begin{itemize}

  \item Perform analysis of the functional and structural connectivity
    of the hippocampal network of patients suffering from Major
    Depressive Disorder and acquire a comprehensive idea about how it
    compares to that of normal individuals from the same
    socio-demographic background.

% to derive conclusions for the sake of validating the statement
% "psychological symptoms and behavioral defects in patients suffering
% from mental health issues are closely related to structural and
% functional changes in specific regions of the brain".


\end{itemize}

% public datasets including those involving
% patient outcomes, relating pathology to cellular expression (RNA seq)
% and genetic phenotypes, among others.

% \begin{itemize}
  % \item The functional and structural connectivity of Hippocampal
    % network in MDD : To study of how structure and function of brain
    % with MDD is different than that of normal. : Study the effect of
    % MDD in hippocampus of patients suffering from MDD
% \end{itemize}

\newpage

\section{Problem Statement}

\subsection{Need For An Imaging Basis}

The diagnosis procedures that are the gold standard for diagnosis of
psychiatric disorders are wholly based on behavioral observations and
patient reported symptoms. There are two most widely established
symptoms that are used to classify these manifestations, one is the
Diagnostic and Statistical Manual of Mental Disorders (DSM) and the
other is International Classification for Diseases (ICD).

Despite each being as widely used as the other, both of these
diagnosis manuals are more like frameworks provide a way of
classifying a psychiatric disorder depending on patterns of behaviour
rather than interpreting the etiology and the physical cause of those
disorders.

This statement alone raises an argument that ``although reliable,
current diagnostic procedures in psychiatry are not entirely valid''.

\iffalse
Reliable in the sense that any trained professional will arrive at
the same diagnosis for each patient.

Valid in the sense that it reflects the underlying psychological
and biological commonalities and differences among different
disorders to a certain extent. Validity continues to be more
difficult to achieve.
\fi

Let us take an example of the diagnostic procedure involved in the
diagnosis of Major Depressive Disorder. The DSM-V, published in 2013,
is the most up-to-date manual and is based upon the work of expert
study groups and makes use of large sets of data. According to the
DSM-V, for a person to be classified as "suffering from Major
Depressive Disorder", \textit{he/she must report with either depressed
mood or anhedonia (inability to feel pleasure in normally pleasurable
activities) along with four out of eight additional symptoms}.

This makes it totally possible for 2 distinct individuals who do not
share a single symptom in common and yet receive treatment (or
medication) for MDD.

Furthermore, the current diagnostic procedures such as the DSM-V are
not perfect. For example, impulsivity, emotional lability (the
property of changing rapidly), and difficulty with concentration each
occurs in more than one disorder.

Now, the fact that, different exemplars of the same category can share
no symptoms and that the exemplars of two different categories may
share common symptoms, raises questions about the validity of the
current diagnostic procedures in psychiatry.

In addition to that, some other medical conditions such as thyroid
disease, brain tumors, vitamin deficiency can mimic depression like
symptoms. Therefore, a may also have to be conducted in order to rule
out some other medical condition that may be causing depressive
symptoms. For instance, a blood test might be done to ensure the
symptoms are not due to thyroid related issues.

Taking the above mentioned arguments into consideration it is crystal
clear that there is a need for an imaging basis for the diagnosis of
mental disorders.

\newpage
\subsection{Reverse Inference Fallacy}

In present day and world, a variety of imaging modalities such as
ultrasonography, x-rays, computed tomography, MRI, SPECT, PET,
fluoroscopy, etc are being implemented for a large number of purposes,
most of them include clinical diagnosis of various diseases and the
others include research. Now, while some of these imaging modalities
such as MRI, SPECT and PET are indeed being used for research that
involve diagnosis of psychiatric disorders, they are yet to be
implemented for the actual diagnosis of mental disorders.

There exists thousands of published research studies using functional
neuroimaging methods such as SPECT, PET, and fMRI that revolve around
diagnosis of mental disorders. However, findings from brain imaging do
not appear amongst the diagnostic criteria; aside from its use to
identify potential physical injury or tumours, neuroimaging is not
used in diagnostic procedure in psychiatry.

Now, at first glance it might seem quite unusual and wrong and foolish
that such advanced imaging techniques are not being used for diagnosis
of mental disorders especially after so many researchers have done
studies on it, there is actually quite a good, and as a matter of
fact, quite an important reason behind it.

The reason behind this is the reverse inference fallacy. Most
psychiatric imaging studies involve subjects from only two categories-
patients from a single diagnostic category and people without any
psychiatric diagnosis (healthy individuals), the most that can be
learned from such a study is how brain activation in those with a
particular disorder differs from brain activation in those without a
disorder. This raises a dilemma for the diagnosing clinician, as the
question is not ``does this person have disorder X or are they
healthy?" but rather ``does this patient have disorder X,Y,Z or are
they healthy?" because the pattern of images that distinguishes
patients with disorder X from healthy people may not be unique to X
but shared with a other disorders.

In addition to the reverse inference fallacy, standardization is
another issue which contributes to neuroimaging not yet finding a
place in psychiatric practice. Standardization is relevant in the
sense that protocols for imaging studies differ from study to study,
particularly amongst functional imaging studies.

Findings on the patterns of activation acquired in studies of
psychiatric patients depends strongly on the task being performed by
the subjects and the statistical comparisons made by the researcher
afterwards. Such findings are pretty much incomplete unless they
include the information about what task evoked the activation in
question: whether the patient wa resting, processing an emotional
stimuli, resisting emotional stimuli or engaged in some other task?
Therefore the fact that imaging study's conclusions are relative to
the tasks performed adds further complexity to the problem of
consistently discriminating patterns of activation of healthy and ill
subjects.


A statistical approach to image analysis makes it possible to
discover, \underline{spatial and temporal} patterns that correspond to
performance of specific tasks and specific diagnoses. Such statistical
methods have only been begun to be applied to clinical disorders but
show promise for increasing the ``specificity'' of brain imaging
markers for mental illness.

Now, in the word of technology that is advancing day in day out,
the scope seems promising.

- development (more sophisticated) methods of image analysis may hold
  promise discerning the underlying differences among the many
  disorders that feature similar regional abnormalities

\include{random dir/literature-review}
\newpage

\section{Feasibility Study}

\begin{itemize}

  \item This project will focus in analysis of the functional and
    structural connectivity of the hippocampal network of patients
    suffering from Major Depressive Disorder.

  \item Although the approach of neurological study is new in Nepal,
    various research has been conducted worldwide, so we can use the
    public data/information present in internet for our project.

  \item There isn't many components required for the project and can
    be carried out in given time frame with low budget.

  \item The research and study can be done pretty much virtually at
    home, which is much more favorable during pandemic moment.

  \item Hence, the feasibility study to conduct this proposed project
    is positive and supportive.

\end{itemize}

\section{Methodology}

Based on resting-state functional magnetic resonance imaging data,
this project will attempt to investigate the functional connectivity
changes in the hippocampal network of 30 MDD patients and 30
well-matched healthy controls.

\section{Cost Estimations}
\section{Time Frame \& Proposed Work Flow}
\section{Conclusion}

Previous studies indicated discrepant functional connectivities
between MDD patients and HC. However, it is unknown whether these
connectivities can be used as diagnostic biomarkers of MDD.18 Indeed,
whether the future diagnostic models built on the functional
connectivity values can improve treatment prediction and clinical
outcome depend on its accuracy performance.

\section{Bibliography \& References}

\end{document}
