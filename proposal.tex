
\documentclass{article}

\usepackage[a4 paper, top=1in, left=1in,right= 1in, bottom=1in,
footskip=0.5in]{geometry}

\usepackage[english]{babel}
\usepackage[utf8]{inputenc}
\usepackage{amsmath}
\usepackage{graphicx}
\usepackage{wrapfig}
\usepackage{enumitem}
\usepackage{tocloft}
\usepackage{listings}
\usepackage{filecontents}
\usepackage{verbatim}
\usepackage{eurosym}
\usepackage[export]{adjustbox}
\usepackage[hidelinks,linktoc=all]{hyperref}
\usepackage{xcolor}
\usepackage{float}
\usepackage[backend=biber]{biblatex}
\usepackage{acronym}

\addbibresource{references.bib}

%% Configs:

\setlength{\parindent}{0em}
\setlength{\parskip}{1.5em}

\graphicspath{{.img/}}

\newcommand{\HRule}{\rule{\linewidth}{0.1mm}}


\begin{document}

\begin{titlepage}
\begin{center}

  \textsc{\huge Purbanchal University}\\[1cm]
  {\huge College of Biomedical Engineering and Applied Sciences}\\[1cm]
  % \textsc{\large \bfseries Tissue Device Interactions}\\[0.8cm]
  % {\large [BEG 3B2 BM]}\\[0.5cm]
  \includegraphics[width=0.4\textwidth]{../cbeas-logo.png}\\[1cm]

  \color{red} \HRule \\[0.4cm] \color{black}
  {\huge \bfseries Seed-based Analysis of Functional Connectivity of
  Hippocampal Network of People Suffering from Clinical Depression}\\[0.2cm]
  \color{red} \HRule \\[2cm] \color{black}

% Author and supervisor
\textbf{
  \Large Proposed By:\\[0.2cm]
\Large Lucky Chaudhary [A27]\\ Namrata Tamang [B3]\\ Nikin Baidar
  [B4]\\ Nilima Sangachchhe [B5]\\ Shashwot Khadka [B18]\\ Sneha
  Khadka [B22]\\Suhana Chand [B23]\\[1cm]}
% \Large Supervisor: \\[0.2cm]
% \Large Prof. Alaka Acharya??}
\vfill
{\Large June 30, 2021}

\end{center}
\end{titlepage}

\pagenumbering{roman}
\clearpage
\setcounter{page}{1}

% preface
\begin{center}
 \textbf{\large Preface}
\end{center}
\newpage

% abstract
\begin{center}
  \textbf{\large Abstract}
\end{center}

The absence of biological markers makes it exceptionally difficult for
neurologists to diagnose a person with a mental disorder. Currently,
diagnosis of mental disorders is based on behavioral observations and
patient-reported symptoms and the Diagnostic and Statistical Manual of
Mental Disorders (DSM) classification. Although there have been
thousands of studies revolving around the implementation of various
imaging modalities for deciphering the etiology and the physical cause
of several mental disorders, the findings from these studies do not
appear amongst the diagnostic criteria. Meaning that the findings from
these studies are not used for diagnosis purposes. A critical barrier
to the clinical translation of many findings is the reverse inference
fallacy as neurological disorders are multifaceted and are influenced
by more than one factor and neuroimaging results can be heavily
influenced by external factors such as patient movement and
instrumental artifacts. However, neuroimaging for diagnosis of mental
disorders seems promising in the future and as a matter of fact, a
bold (choose correct word here) minority have already started to
implement neuroimaging techniques such as fMRI, SPECT, PET for the
diagnosis of psychiatric disorders. Nevertheless, there is no solid
molecular or imaging basis that is widely accepted for the assessment
of mental disorders. Here in the proposed research we plan to evaluate
functional network connectivity of the hippocampus in patients with
Major Depressive Disorder based on their MR images and investigate the
changes in the brain network compared to that of healthy controls
matched according to age and gender.

\newpage

% abstract
\begin{center}
  \textbf{\large Abbreviations}
\end{center}

\begin{acronym}
    \acro{ACC}{Anterior Cingulate Cortex}
    \acro{AFNI}{Analysis of Functional Neuroimages}
    \acro{ASN}{Anterior Salience network}
    \acro{BDI}{Beck Depression Inventory}
    \acro{BOLD}{Blood Oxygen Level Dependent}
    \acro{CA}{Cornu Ammonis}
    \acro{DG}{Dentate Gyrus}
    \acro{DMN}{Default Mode Network}
    \acro{DSM}{Diagnostic and Statistical Manual of Mental Disorders}
    \acro{ECN}{Executive Control Network}
    \acro{EEG}{Electroencephalography}
    \acro{fMRI}{Functional Magnetic Resonance Imaging}
    \acro{HC}{Healthy Controls}
    \acro{HPA}{Hypothalamic Pituitary Adrenal}
    \acro{ICA}{Independent Components Analysis}
    \acro{ICD}{International Classification for Diseases}
    \acro{MDD}{Major Depressive Disorder}
    \acro{MFC}{Medial Frontal Cortex}
    \acro{MFG}{Middle Frontal Gyrus}
    \acro{MR}{Magnetic Resonance}
    \acro{OFC}{Orbitofrontal Cortex}
    \acro{PET}{Positron Emission Tomography}
    \acro{PMC}{Premotor Cortex}
    \acro{rFPDD}{Recurrent Familial Pure Depressive Disorder}
    \acro{ROI}{Region of Interest}
    \acro{rs}{Resting State}
    \acro{rsFC}{Resting-state Functional Connectivity}
    \acro{SCA}{Seed-based Correlation Analysis}
    \acro{SPECT}{Single-Photon Emission Computed Tomography}
    \acro{sMRI}{Structural Magnetic Resonance Imaging}
    \acro{SNRI}{Serotonin and Norepinephrine Reuptake Inhibitors}
    \acro{SSRIs}{Selective Serotonin Reuptake Inhibitors}
    \acro{vPFC}{Ventrolateral Prefrontal Cortex}
\end{acronym}

\newpage

% table of contents
\thispagestyle{empty}
\tableofcontents
\addcontentsline{}{}{}
\addtocontents{toc}{~\hfill\textbf{Page}\par}
\newpage

\pagenumbering{arabic}
\clearpage
\setcounter{page}{1}

\section{Introduction}

\subsection{Major Depressive Disorder}

Major Depressive Disorder, generally abbreviated as MDD, is one of the
most common and serious mental disorders. MDD is also referred to as
clinical depression, or just depression as well. MDD can be
characterized by an array of distinct symptoms; persistent feeling of
sadness, feelings of low self-worth and guilt, and overall reduced
ability to take pleasure from activities that previously were
enjoyable are a few symptoms prevalent in MDD. Although, the exact
symptoms of depression may vary from person to person, depending on
their upbringing and various socio-demographic variables such as age,
sex, religious affiliations, employment, income, etc for an individual
to be classified as “suffering from MDD”, five out of ten symptoms,
one from a set of two and additional symptoms from another set of
five, must be present during a span of two weeks \cite{whodepression}.
In addition to the symptoms that may be prevalent in a person
suffering from depression, there can be morphological differences in
several brain regions, including the frontal and temporal lobes. On
top of that, individuals suffering from MDD also have abnormal
functional connectivity. According to the World Health Organization,
more than 264 million people of all ages suffer from depression
worldwide \cite{whodepression}. Fortunately, there are effective
psychological and pharmacological treatments for moderate to severe
depression. The pharmacological treatment includes medications such
as, SSRIs and SNRI which are the two most commonly prescribed
antidepressants \cite{resting}. The psychological treatments include
psychotherapy and electroconvulsive therapy depending on the severity
of the depression. These treatments can take a few weeks or much
longer.

\subsection{Brain Networks}

A brain network, on a large scale, can be defined as a collection of
brain regions working together to produce a specific function. Brain
networks can be identified at different resolutions, therefore there
is no universal atlas of brain networks that fits all circumstances.
However, based on converging evidence from related studies, six
large-scale, core brain networks that are most widely accepted due to
their stability:

\begin{enumerate}[nosep]
  \item Default Mode Network
  \item Salience Network
  \item Dorsal Attention Network
  \item Frontoparietal Network
  \item Sensorimotor Network
  \item Visual Network
\end{enumerate}

There are more subsets of these six networks such as the limbic,
auditory, right/left executive, cerebellar, spatial attention,
language, lateral visual, temporal and visual perception/imagery. An
emerging paradigm in neuroscience is that cognitive tasks are
performed not by individual brain regions working in isolation but
rather by brain networks consisting of several discrete brain regions
that are said to be ``functionally connected''. The functional
connectivity of brain networks can be acknowledged through
(statistical) analysis of images acquired through a variety of
techniques such as the fMRI, EEG, PET or SPECT
\cite{wikibrainnetworks}.
\newpage

\subsection{Resting State Functional Connectivity}

Functional connectivity can be defined as the temporal correlation
between spatially remote neurophysiological events, expressed as
deviation from statistical independence across these events in
distributed neuronal groups and areas. The results of the study
connected by Bharat B. Biswal et. al. suggests that while variations
in blood flow might contribute to functional connectivity maps, BOLD
signals play a dominant role in the mechanism that gives rise to
functional connectivity in the human brain \cite{biswalsimultaneous}.
During resting conditions, our brain remains functionally and
metabolically active. The fact that brain remains ``metabolically
active" means there will be consumption of oxygen which results in
fluctuations in the BOLD signal. Resting state functional connectivity
can be defined as the correlation patterns in the spontaneous
fluctuations of BOLD signal in the absence of any stimulus or task.
\cite{frontiers:rsfc}.

Resting-state functional connectivity measures
the temporal correlation of spontaneous Blood Oxygen Level Dependent
(BOLD) signals among spatially distributed brain regions, with the
assumption that regions with correlated activity form functional
networks.

There are two methods that are most commonly used to examine
functional connectivity:

\begin{itemize}[nosep]
  \item Seed-based Correlation Analysis (SCA) and
  \item Independent Components Analysis (ICA)
\end{itemize}

In seed-based approaches, activity is extracted from a specific region
of interest and correlated with the rest of the brain. In contrast,
ICA does not begin with pre-defined brain regions. It is a
multivariate, data- driven approach that deconstructs fMRI time-series
data throughout the brain into separate spatially independent
components. The resting-state fMRI study produces reliable and
reproducible results, and several features of resting-state fMRI makes
it favorable for investigating the functional correlation of various
brain regions in psychiatric and neurological disorders. First,
compared to the modular representations of traditional fMRI,
functional connectivity provides a broader network representation of
the functional architecture of the brain. Second, the absence of an
explicit task eases the cognitive demand of the fMRI environment,
thereby eliminating the problem of whether or not to match groups on
task performance and allowing researchers to investigate under-studied
populations, including infants and cognitively impaired individuals.
Finally, the relatively standard manner in which resting-state fMRI
data are acquired makes it ideal for multi-site investigations and
data sharing.\cite{resting}

\subsection{Neuroimaging}

Neuroimaging or brain imaging is the use of various imaging modalities
to either directly or indirectly image the structure, function, or
pharmacology of the nervous system. Current neuroimaging techniques
reveal both form and function. They reveal the brain's anatomy,
including the integrity of brain structures and their
interconnections. Neuroimaging can be divided into two broad
categories:
\vspace{-10pt}
\begin{enumerate}

  \item \textbf{Structural Imaging}, which deals with the structure of
    the nervous system and the diagnosis of gross (large scale)
    intracranial disease (such as a tumor) and head injury.

  \item \textbf{Functional imaging}, which is used to diagnose
        metabolic diseases and lesions on a finer scale (such as
        Alzheimer's disease) and also for neurological and cognitive
        psychology research and building brain-computer interfaces
        \cite{neuroimaging}.

\end{enumerate}

\vspace{-10pt}
\enlargethispage{\baselineskip}
Functional Magnetic resonance imaging (fMRI) is a modern technique of
imaging, which is a powerful non-invasive and safe tool used for the
study of the function of the brain based on the measure of the brain
neural activation. The fMRI can localize the location of activity in
the  brain  which  is  caused  due  to sensory stimulation or
cognitive  function. In the clinical setting, fMRI allows the
researchers to study how healthy brain functions, how different
diseases affect the brain functions, how brain functions altered due
to disease or injury can be restored, and how how drugs can control
the disease's effect on brain activity \cite{fMRI}.

\newpage

\section{Objectives}

The objectives of the proposed project are as follows:

\subsection{Specific Objectives}

\begin{itemize}

  \item Deploy computational tools, and implement image processing
        strategies for the exploration of MR image datasets of the
        human brain using AFNI.

  \item Explore data visualization tools, with emphasis on displaying
        functional brain networks.

  \item To perform Seed-based Analysis (SCA) to explore functional
        connectivity within the brain based on the time series of a seed
        voxel or Region of Interest (ROI).

\end{itemize}

\subsection{General Objectives}

\begin{itemize}

  \item Perform analysis of the functional connectivity of the
      hippocampus in patients suffering from Major Depressive Disorder
      and acquire a comprehensive idea about how it compares to that
      of normal individuals of the same age group.

\end{itemize}

\newpage

\section{Problem Statement}

\subsection{Need For An Imaging Basis}

The diagnosis procedures that are the gold standard for the diagnosis
of psychiatric disorders are wholly based on behavioral observations
and patient reported symptoms. Current diagnostic procedures involved
do not have an imaging or a biochemical basis \cite{noimagingbasis}
\cite{nobiochemicalbasis}. Two most widely established symptoms are
used to classify these manifestations, one is the Diagnostic and
Statistical Manual of Mental Disorders (DSM) and the other is
International Classification for Diseases (ICD). Despite each being,
widely used as the other, both of these diagnosis manuals are more
like frameworks that provide a way of classifying a psychiatric
disorder depending on patterns of behaviour rather than interpreting
the etiology and the physical cause of those disorders. This statement
alone raises an argument that “although reliable, current diagnostic
procedures in psychiatry are not entirely valid”.

Let us take an example of the diagnostic procedure involved in the
diagnosis of Major Depressive Disorder. The DSM-V, published in 2013,
is the most up-to-date manual and is based upon the work of expert
study groups and makes use of large sets of data. According to the
DSM-V, for a person to be classified as ”suffering from Major
Depressive Disorder”, he/she must report with either depressed mood or
anhedonia (inability to feel pleasure in normally pleasurable
activities) along with four out of eight additional symptoms
\cite{diagnosticbrainimaging}. This makes it possible for two distinct
individuals who do not share a single symptom in common and to receive
similar treatment (or medication) for MDD.

Furthermore, the current diagnostic procedures such as the DSM-V are
not perfect. For example, impulsivity, emotional lability (the
property of changing rapidly), and difficulty with concentration each
occur in more than one disorder. Now, the fact that different
exemplars of the same category can share no symptoms and that the
exemplars of two different categories may share common symptoms,
raises questions about the validity of the current diagnostic
procedures in psychiatry.

In addition to that, some other medical conditions such as thyroid
disease, brain tumors, and vitamin deficiency can mimic
depression-like symptoms\cite{externalfactorsinMDD}. Therefore, a
diagnosis may also have to be conducted in order to rule out some
other medical condition that might be causing depressive symptoms. For
instance, a blood test can be done to ensure the symptoms are not due
to thyroid related issues.

Considering the above mentioned arguments, an imaging basis for the
diagnosis of mental disorders seems like the need of the hour.

\newpage
\subsection{Reverse Inference Fallacy}

In present day and world, a variety of imaging modalities such as
ultrasonography, x-rays, computed tomography, MRI, SPECT, PET,
fluoroscopy, etc are being implemented for a large number of purposes,
most of them include clinical diagnosis of various diseases and the
others include research. Now, while some of these imaging modalities
such as MRI, SPECT and PET are indeed being used for research that
involve diagnosis of psychiatric disorders, they are yet to be
implemented for the actual diagnosis of mental disorders.

There exists thousands of published research studies using functional
neuroimaging methods such as SPECT, PET, and fMRI that revolve around
diagnosis of mental disorders. However, findings from brain imaging do
not appear amongst the diagnostic criteria; aside from its use to
identify potential physical injury or tumours, neuroimaging is not
used in diagnostic procedure in psychiatry.

Now, at first glance it might seem quite unusual and wrong and foolish
that such advanced imaging techniques are not being used for diagnosis
of mental disorders especially after so many researchers have done
studies on it, there is actually quite a good, and as a matter of
fact, quite an important reason behind it.

The reason behind this is the reverse inference fallacy. Most
psychiatric imaging studies involve subjects from only two categories-
patients from a single diagnostic category and people without any
psychiatric diagnosis (healthy individuals), the most that can be
learned from such a study is how brain activation in those with a
particular disorder differs from brain activation in those without a
disorder. This raises a dilemma for the diagnosing clinician, as the
question is not ``does this person have disorder X or are they
healthy?" but rather ``does this patient have disorder X,Y,Z or are
they healthy?" because the pattern of images that distinguishes
patients with disorder X from healthy people may not be unique to X
but shared with a other disorders.

In addition to the reverse inference fallacy, standardization is
another issue which contributes to neuroimaging not yet finding a
place in psychiatric practice. Standardization is relevant in the
sense that protocols for imaging studies differ from study to study,
particularly amongst functional imaging studies.

Findings on the patterns of activation acquired in studies of
psychiatric patients depends strongly on the task being performed by
the subjects and the statistical comparisons made by the researcher
afterwards. Such findings are pretty much incomplete unless they
include the information about what task evoked the activation in
question: whether the patient wa resting, processing an emotional
stimuli, resisting emotional stimuli or engaged in some other task?
Therefore the fact that imaging study's conclusions are relative to
the tasks performed adds further complexity to the problem of
consistently discriminating patterns of activation of healthy and ill
subjects.

Now, in the word of technology that is advancing day in day out,
development (more sophisticated) methods of image analysis may hold
promise discerning the underlying differences among the many disorders
that feature similar regional abnormalities. Nevertheless, the scope
of neuroimaging seems promising for the aid in diagnosis, and
hopefully treatment of psychiatric disorders.
\newpage

\section{Review of Literature}

\subsection{Background}
% opening note:
% background and aim of the review

Efforts are continuously being made in order to discover reliable
biomarkers for the clarification of biological mechanisms that are
involved in psychiatric disorders, identification of subjects at risk
and provide etiology-based treatments. Imaging modalities such as
structural magnetic resonance imaging (sMRI) and functional magnetic
resonance imaging (fMRI) are used to outline brain irregularities over
Major depressive disorder (MDD).

Multiple modalities have been considered for assessment of functional
connectivity of brain networks, but fMRI is the most commonly used
amongst the others. This is because (insert reason a valid here).

% For this reason,
Most studies that have been referred to, whilst writing this review
have utilized fMR-imaging modality to conduct investigations on the
core aspects of functional brain alterations in patients suffering
from MDD.

The goal of this literature review is to gain a comprehensible
knowledge about the associations of various different brain networks
with Major Depressive Disorder and also to acquire a brief overview of
how the brain networks, especially the hippocampal network gets
affected by MDD.

% insert more info on what the project is concerned with
Since this project is mostly concerned with the hippocampal network,
most of the literature will be more or less be related to the temporal
lobe and its structures.

%               __              __                  _
%   ____ ______/ /___  ______ _/ /  ________ _   __(_)__ _      __
%  / __ `/ ___/ __/ / / / __ `/ /  / ___/ _ \ | / / / _ \ | /| / /
% / /_/ / /__/ /_/ /_/ / /_/ / /  / /  /  __/ |/ / /  __/ |/ |/ /
% \__,_/\___/\__/\__,_/\__,_/_/  /_/   \___/|___/_/\___/|__/|__/

% Lucky

% Neuroimaging correlates of depression ???
\subsection{Functional Magnetic Resonance Imaging for the Assessment
of MDD}

People with MDD show distinct functional alterations that differ from
those of healthy individuals. Functional brain alterations can be
found by detecting the activation of specific brain regions. A brain
region become active when there is blood flow in that region, and an
elevated level of metabolism. Now, Unlike structural brain imaging
that captures the anatomical structures present in the brain,
functional brain imaging involves measurement of blood flow and
metabolism to visualize the activation of specific brain regions.
Therefore, functional imaging techniques such as the fMRI indicates
the activation of various different brain regions which makes it quite
convenient to identify what parts of brain are active during a
condition.

fMRI comes in two flavours, one is the resting-state fMRI and the
other is the task-based fMRI. ``Resting-state'' is when a person is
fully awake but isn't performing any particular task that requires
attention and cognition. While many studies are based on the rs-fMRI,
researches believe that the rs-fMRI lacks the linearity and stationary
signals required for the assessment of MDD.

% The default mode network, which comprises the ventromedial prefrontal
% cortex, posterior cingulate cortex, and precuneus, is primarily active
% when no task is being performed and the brain is passively resting.
% However, for tasks that require attention and cognition the DMN
% becomes pretty much inactive.

There have been limited reports of functional alterations in the
temporal lobe with the use of pure rs-fMRI. Nonetheless, in one study,
treatment resistant patients with MDD showed increased levels of
hippocampal activation during loss events.

Yu et al., in their reasearch showed functional alterations in the
activity of the right hippocampus, right para-hippocampal gyrus, left
amygdala and the entire caudate nucleus, which ultimately suggests
that the temporal lobe and various structures in the temporal lobe,
such as the hippocampus might have an important pathophysiology of
MDD.  Therefore, additional studies are needed to determine the
relevance of these findings.

In another study, the duration of MDD was directly associated with
hippocampal volume loss in women with MDD.

It was also found in some task-based fMRI studies that the
presentation of sad faces led to increased activation of left
hippocampus, amygdala and para-hippocampal gyrus.

Stoyanov and colleagues found out that there is a weak correlation
between medial frontal cortex (MFC) and MDD subjects. In addition,
they also made an implication that the pathophysiology of MDD was
because of the activation in anterior thalamus, hippocampus and
para-hippocampal gyrus areas.

\iffalse

% what are all the structures mentioned here a part of ? brain
% networks and how do
% they relate to hippocampus/ our study? :

Patients with MDD were prone to have an increased activity in the
medial prefrontal cortex (MPFC)/anterior cingulate cortex (ACC) areas
with diminished activity in posterior cingulate cortex(PCC)/precuneus
and bilateral angular gyrus areas.

% Some of the articles showed: what articles did ?

1. Increased connectivity in depression in the bilateral dorsomedial
   prefrontal cortex

2. Increased functional network connectivity in the
   thalamus, subgenual cingulate, the precuneus and orbitofrontal
   cortex (OFC)

3. Decreased influence from the anterior insula to the
   middle frontal gyrus was found in medicated subjects with MDD

4. The relation that the right anterior insula has on depression
   pathophysiology was confirmed through the positive correlation
   between hippocampal node activation and severity of depression

\fi

% Nilima

\subsection{Resting State Functional Connectivity of Hippocampal Networks}

Various studies have focused on the abnormal functional connectivity
of several brain networks in the patients with MDD.

Studies have shown that MDD not only shows associations with regional
deficits, but also with abnormal functional integration of distributed
brain regions. A number of brain regions with abnormal activities in
the resting-state have been identified to be associated with MDD, such
as para-hippocampal gyrus, prefrontal cortex, cingulate gyrus,
fusiform gyrus, and thalamus. Moreover, disruptions in functional
connectivity has been observed between specific pairs of regions in
MDD through functional connectivity analyses which may or may not
include seed-based correlation analysis.

Greicius et al. (2007) used the independent component approach (ICA),
selecting a set of regions with shared fMRI signal fluctuations and a
high degree of spatial similarity to the DMN, and reported increased
connectivity with the thalamus and the subgenual ACC in depression.

Many studies have found that the hippocampus, which is complex
structure embedded deep into the temporal lobe, plays an important
role in MDD. Hippocampus can be subdivided into 3 sub-structures, and
theses structures are enumerated below:

\begin{enumerate}[nosep]
    \item Cornu Ammonis (CA)
    \item Dentate Gyrus (DG)
    \item Subiculum
\end{enumerate}

Various studies have been performed with each of these sub-structures
as the seed or the region of interest. The findings from some of the
studies that are relevant to this project are listed below:

\begin{itemize}

    \item Increased connectivity in the left premotor cortex (PMC) and
        reduced connectivity in the right insula with the CA seed
        region.

    \item Increased connectivity was reported in the left
        orbitofrontal cortex (OFC) and left ventrolateral prefrontal
        cortex (vPFC) with the DG seed region.

    \item The subiculum seed region revealed increased connectivity
        with the left premotor cortex (PMC), the right middle frontal
        gyrus (MFG), the left ventrolateral prefrontal cortex (vPFC)
        and reduced connectivity with the right insula.

\end{itemize}

Furthermore, a region-of-interest based correlation analyses performed
in rs-fMRI showed FC with the hippocampus in limbic system, sub
cortical areas, temporal lobe, medial and inferior prefrontal cortex,
while at the same time, negative FC was in bilateral prefrontal
cortex, parietal and occipital cortex and the cerebellum.

\enlargethispage{\baselineskip}
In addition to that, many researches have implicated abnormalities in
the prefrontal-hippocampus neural circuitry in patients suffering from
MDD. fMRI studies have also found abnormal hippocampal activation as
well as abnormal functional connectivity of prefrontal-hippocampus
circuitry in adults who were suffering from MDD. Moreover, Peng et.
al. in one of their recent studies reported decreased rsFC between
hippocampus and insula in medication-resistant adult patients.

The hippocampus has been proven to play an important role in memory
and emotion processing. Functional abnormalities of the hippocampus in
adult MDD have been consistently reported in several fMRI studies.
According to an fMRI study, decreased brain activity in the
hippocampus was reported in depressive patients.

Similarly, the hippocampus and amygdala of MDD patient’s showed an
overlapping pattern of reduced FC to the dorsomedial-prefrontal cortex
and fronto-insular operculum. Both of these regions are known to
regulate the interactions among intrinsic networks (i.e., default
mode, central executive, and salience networks) that are disrupted in
MDD.

A few postmortem studies have found decreased cellular density in the
hippocampus, including one study that showed patients with MDD have
fewer anterior dentate gyrus granule cells than control subjects.
However, functional imaging studies at this resolution in patients
with MDD are lacking.

For several reasons, researchers have focused on the role of the
hippocampus in depression. The  hippocampus is involved in the
regulation of the hypothalamic pituitary adrenal (HPA)-axis, which is
responsible for production of stress-related glucocorticoids such as
cortisol. In this context, depressed individuals have been found
consistently to report high levels of stress, which is reflected
biologically in elevated rates of hypercortisolemia and disturbed
HPA-axis functioning. Moreover, depressed patients have also been
found to be characterized by difficulties in hippocampal-dependent
learning  and  memory.Also, Problems  can occur  when excessive
amounts  of  cortisol  are  sent to  the  brain due  to  a stressful
event  or  a chemical imbalance in the body.

\subsubsection{Overview of Hippocampal Circuitry and its Functions}

Hippocampus is a relatively simple one that can be related to the
functional requirement of episodic memory and more specifically to the
storage and retrieval of memory.

The hippocampus is part of the hippocampal formation which includes
the dentate gyrus, Para hippocampal gyrus, and hippocampal gyrus.

The hippocampal gyrus contains areas such as the entorhinal cortex and
subiculum, which are both vital in the flow of information through the
hippocampus.

The hippocampus is divided into CA1 to CA4 regions which stands for
cornu ammonis. The CA1 has an important role in memory due to the high
levels of NMDA glutamatergic receptors located in the CA1 Schaffer
collateral neurons.

The synaptic plasticity of these neurons heavily relies on long-term
potentiation (LTP) induction to allow the strengthening of declarative
memory which includes two types of implicit memory such as episodic
and semantic memory such as facts and events.

\newpage
% Namrata

\subsection{Important FCs for Diagnosis of MDD}

Default mode network (DMN), Anterior salience network (ASN) and
Executive control network (ECN) are the brain networks linked with
clinical depression. Increased functional connectivity within the DMN
is primarily associated with depression. At the same time, it is found
that the DMN-ECN and the DMN-ASN pairs have less interactions or
connectivity during episodes of depression. Within the ECN, the
functional connectivity may be excessive or deficient. Furthermore,
the ECN in depressed women is correlated with negative self-directed
thoughts and the ECN-DMN functional connectivity is related to
rumination.  Researches have shown that ASN which includes main
emotional areas maybe over, under or normally connected in depression.
Depression is also related with the impaired functional connectivity
of other brain networks besides the ones mentioned above.

In addition to that the Posterior Cingulate Cortex which is a part of
the DMN has shown significant relationship with the hippocampal
network, albeit this is not just specific to MDD.

% However, all of these findings seem inconsistent as the results vary
% from one research to another. Evidence for increased or decreased
% functional connectivity of the three major networks mentioned is
% needed.

Hindawi Neural Plasticity et. al., on comparing the functional
activity within DMN, found that there was decreased functional
connectivity within DMN network in depressed people, which
contradicted with the most researches that have been conducted. This
may be because the patients involved in their study had mild to
moderate depression, which showed cognitive similar features as in
depression (MDD) like rumination and control deficits, but lacked
neural markers present in typical serious condition like MDD.

Significant rs-fMRI differences between groups were identified in
multiple clusters in the DMN and ECN. Greater positive connectivity
within the ECN and between ECN and DMN regions was associated with
poorer episodic memory performance in the group of healthy individuals
but better performance in the MDD group. Greater connectivity within
the DMN was associated with better episodic and working memory
performance in the Non-Depressed group but worse performance in the
MDD group.

These results provide evidence that cognitive performance in MDD may
be associated with aberrant functional connectivity in cognitive brain
networks and suggest patterns of alternate brain function that may
support cognitive processes in MDD.

Results also showed that the DMN–left fronto-parietal network is the
pair discriminating between healthy and depressed people to the
highest degree. According to Davidson and Heller models, left
prefrontal activity is related to positive emotions and motivations,
while right corresponds to negative emotions and withdrawal of
motivation.

Relatively active right prefrontal area and idling left
prefrontal cortex together may be a neurophysiological signature of
depression. Therefore, coupling of left prefrontal with DMN denotes
its passivity and that there's is less approach motivation, less happy
mood which is one of the most important depression related signs.

% concluding note:

Decreased functional connectivity between left fronto-parietal network
and subsystems of the DMN can be seen in fMR images of depressed
patient.

Many studies that have been published showed increased functional
connectivity within DMN in depression, which was found to be
contradiction with one particular research paper which may be due to
sample differences like severity of the disorder, age group,and other
demographics.

\newpage
\subsection{Structural Changes associated with MDD}

The latest research shows that the size of specific [brain](brain)
regions can decrease in people who experience depression. Researchers
continue to debate which regions of the brain can shrink due to
depression and by how much.

Hippocampus, thalamus, amygdala and the frontal prefrontal cortices
are the regions that have the most shrinkage. The amount by which,
these areas shrink is linked to the severity and the length of the
depressive episodes. In the hippocampus, for example, noticeable
changes can occur anywhere from 8 months to a year.

It was found in a study that people experiencing their first
depressive episode had a normal hippocampus size but as the number of
episodes of depression a person had would increase, the greater the
reduction in hippocampus size. It has been widely reported that there
is a significant reduction in hippocampal volume in depression
patients.  This situation was found in both adult and adolescent
depressed patients, whether they were in their first or recurrent
depressive episodes. A recent study reported that, in female patients
with recurrent familial pure depressive disorder (rFPDD), volumetric
reductions of the right hippocampal body and tail were significantly
larger than those of the left, while the whole brain volume was
approximately equal to that of healthy subjects.

There is evidence that stress via the hypothalamic-pituitary-adrenal
axis can result in elevated glucocorticoid levels in patients with
depression and can act on the glucocorticoid receptors in the
hippocampus. Thus, hippocampal atrophy occurs as a result.

Reduced gray matter volume and reduced functional activity in the
hippocampus would lead to negative emotion and the inability of
cognitive processing in depressive patients.  Depression can also
decrease neuronal dendrite branching and plasticity in the
hippocampus.

In addition, depression can trigger activation of the
hypothalamic-pituitary-adrenal axis, increase level of
corticosteroids, and down regulate hippocampal neurogenesis.
Depression makes changes in hippocampal volumetric changes,
hippocampal neurogenesis, and apoptosis of hippocampal neurons.

\subsection{Reverse Inference Fallacy in MDD}

The diagnosis of MDD relies heavily on patients for symptom recall.
Like in many other mental disorders there is no specific physical
symptom in MDD, therefore neurologists are unable to deduce MDD based
on conventional MRI.

Moreover, different mental disorders can produce similar structural or
functional brain alterations which puts the neurologist in a much
greater dilemma as they are unable to determine whether a specific
alteration in the brain is attributed to MDD-alone.  So, the reverse
inference is invalid as structural or functional activity patterns
from MRI cannot be used to diagnose a patient with a specific
neurological disorder.

Researchers, over the past few decades have been attempting to develop
a “bridge”, such as novel biomaterial, high resolution and
multi-modality imaging technique, artificial intelligence, novel
nanomaterials and quantitative electrical signal acquisition
technologies to overcome reverse inference fallacy that hinders the
study of MDD.

\subsection*{Conclusion of the Literature Review}

Evidence shows that major depressive disorder (MDD) patients at
resting-state brain connectivities are aberrant compared with healthy
controls (HC). Abnormal resting-state functional connectivities of
distributed brain networks are believed to contribute to the MDD
illness process.

And after a thorough review of literature, it seems as if structural
MRI and fMRI look promising for providing excellent and reliable
indexes for the aid in the diagnosis and ultimately treatment of MDD.

Once it over-comes the afore mentioned hurdles, MRI may become a
clinical decision support tool aimed to reduce unsuccessful treatments
and improve treatment efficacy and efficiency.

Reliable, reproducible and valid conclusions must be derived from
these types of studies for imaging modalities like fMRI to not only
aid in the diagnosis but also to optimize patient care, reduce
treatment resistance and shorten the duration of illness.

\section{Feasibility Study}

The following points addresses the feasibility of the proposed study:

\begin{itemize}

  \item This project will primarily focus on analysis of the
    functional connectivity of the hippocampal network of patients
    suffering from Major Depressive Disorder.

  \item Although the approach of neurological study is new in Nepal,
    thousands of research has been conducted worldwide that involves
    similar approaches for the exploration of functional connectivity.


  \item Data will be acquired form a public database, which is
    available online for the sake of this project.

  \item There are not many materials or components required for the
    project. We will be using open-source softwares such as Analysis
    of Functional Neuro Images (AFNI) and the Linux operating system,
    this adds to the feasibility of this project.

  \item Furthermore, all the work involved in this project can pretty
    much be conducted virtually at home, therefore this eliminates the
    obstacles that may arise due to the on going pandemic.

  \item Hence, the feasibility study to conduct this proposed project
    is positive and supportive.

\end{itemize}

\newpage

\section{Methodology}

Based on resting-state functional magnetic resonance imaging data,
this project will attempt to investigate the functional connectivity
changes in the hippocampal network of 30 MDD patients and just as many
well-matched healthy controls.

\subsection{Methods and materials}

We plan to employ resting-state functional MRI (rs-fMRI) to
investigate topological changes of the functional connectome in
patients with MDD. Our plan is to collect data from (NUMBER) MDD, and
(NUMBER) healthy controls (HC) to study alterations in functional
connectivity of hippocampus regions, and to explore their relationship
with memory and emotional behaviors.

\subsection{Data Acquisition}

For the data acquisition part, the \textit{SRPBS Multiorder MRI
Dataset} was used. The public dataset that we will be using contains
the 3T MRI images data from 1410 participates collected at 11 sites.

Statistical analysis methods such as the T-test and the P-test will be
applied on this big data to narrow it down to just 30 participants
including (NUMBER) patients with MDD and (NUMBER) demographically
matched healthy controls (HC) based on age and sex.

Importantly, we plan to choose (number) right handed patients from
each group with MDD and HC. Here, we engaged MDD patients with BDI
(Beck depression Inventory) index more than 30.

Since we will be acquiring the data from large scale public dataset,
diagnosed by neurologist, this assures our research to be tilted more
to accuracy and efficiency. All the participants, extracted from the
public data set, have underwent a standardized clinical evaluation
protocol, which included a general and neurological examination, which
will make our research more feasible.

\subsection{Image Pre-processing}

Before statistical analysis, it is necessary to convert the raw fMRI
data to a form that is understandable by the software and also to
improve the signal quality of the data obtained from the MRI scanner,
which includes artifact detection, baseline correction, realignment,
movement correction, co-registration, normalization, and smoothing.

The fMRI data will be preprocessed using AFNI. AFNI is an open source
software created an maintained by Robert W. Cox at NIH.

Pre-processing involves spatial or temporal filtering of the fMRI data
and improving the image resolution. Seed based resting-state
functional connectivity (rsFC) analysis of hippocampal network will be
performed such that for each seed region, rsFC will be calculated as
the correlation between its \textit{mean time} course and the time
course of every voxel in the brain. Then we will submit the rsFC
results for each seed region to a 2 x 3 (hemisphere x group)
mixed-design analysis of variance with age and head motion included as
covariates. The final rsFC results, between the MDD patients and HC
will be compared using statistical approach.

% The goal of preprocessing is to eliminate different kinds of artifacts
% such as motion correction.

\subsection{Functional connectivity analysis}

Once the rs-fMR images from the public data set has been
pre-processed, Seed- based rsFC analyses will be performed by
implementing statistical methods.

Specifically, we plan to assess functional connectivity between
various regions of hippocampal area using fMRI. To assess functional
connectivity in the brain region, Resting-state analyses, that is,
time-series correlations in BOLD fMRI data acquired in a task-free
state will be used.

A statistical approach to image analysis makes it possible to
discover, \underline{spatial and temporal} patterns that correspond to
performance of specific tasks and specific diagnoses. Such statistical
methods have only been begun to be applied to clinical disorders but
show promise for increasing the ``specificity'' of brain imaging
markers for mental illness.

\newpage

\section{Cost Estimations}
\section{Time Frame \& Proposed Work Flow}
\section{Conclusion}

Previous studies indicated discrepant functional connectivities
between MDD patients and HC. However, it is unknown whether these
connectivities can be used as diagnostic biomarkers of MDD.18 Indeed,
whether the future diagnostic models built on the functional
connectivity values can improve treatment prediction and clinical
outcome depend on its accuracy performance.

While clearly not sufficient to provide a detailed understanding of
the complex and changing functional connectivity of the brain which
can make actual diagnosis of MDD, this project will lay the
foundations for further research and development.

\newpage

\printbibliography
\end{document}
